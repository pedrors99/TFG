\section{Aplicaciones}

Con el fin de motivar al lector a investigar más sobre las pruebas de conocimiento cero, este capítulo contiene algunas aplicaciones interesantes.

\begin{itemize}
    \item Pagos anónimos: Los pagos con tarjeta de crédito a menudo son visibles para varias partes, incluido el proveedor de pagos, los bancos y otras partes interesadas (por ejemplo, autoridades gubernamentales). Si bien la vigilancia financiera tiene beneficios para identificar actividades ilegales, también socava la privacidad de los ciudadanos comunes.
    
    Las criptomonedas estaban destinadas a proporcionar un medio para que los usuarios realizaran transacciones privadas entre pares. Pero la mayoría de las transacciones de criptomonedas son visibles abiertamente en cadenas de bloques públicas. Las identidades de los usuarios a menudo son seudónimas y están vinculadas intencionalmente a identidades del mundo real (por ejemplo, al incluir direcciones ETH en perfiles de Twitter o GitHub) o pueden asociarse con identidades del mundo real utilizando análisis de datos básicos dentro y fuera de la cadena.

    Al incorporar tecnología de conocimiento cero en el protocolo, las redes de cadena de bloques centradas en la privacidad permiten que los nodos validen las transacciones sin necesidad de acceder a los datos de la transacción.
    
    \item Autenticación: El uso de servicios en línea requiere que demuestre su identidad y derecho a acceder a esas plataformas. Esto a menudo requiere proporcionar información personal, como nombres, direcciones de correo electrónico, fechas de nacimiento, etc. También es posible que deba memorizar contraseñas largas o correr el riesgo de perder el acceso.
    
    Sin embargo, las pruebas de conocimiento cero pueden simplificar la autenticación tanto para las plataformas como para los usuarios. Una vez que se ha generado una prueba ZK utilizando entradas públicas (por ejemplo, datos que acrediten la membresía del usuario en la plataforma) y entradas privadas (por ejemplo, los detalles del usuario), el usuario puede simplemente presentarla para autenticar su identidad cuando necesite acceder el servicio. Esto mejora la experiencia de los usuarios y libera a las organizaciones de la necesidad de almacenar grandes cantidades de información de los usuarios.

    \item Más de 18 ZKRP: Se puede usar para demostrar que alguien tiene más de 18 años sin revelar su edad exacta. Así es posible permitir que la persona consuma algún servicio sin exigirle que muestre documentos en papel, que contienen más información de la necesaria para la validación de la edad.

    En esta situación es importante contar con una parte de confianza para generar un compromiso, que acredite que la información contenida en el mismo es correcta. La persona no puede generar el compromiso por sí misma porque un usuario malintencionado podría utilizar datos falsos para probar la declaración deseada, aunque los datos reales no respeten esa propiedad.
    
    \item Know Your Customer (KYC): ZKRP permite validar que una determinada información privada pertenece a un intervalo numérico. Esta propiedad se puede utilizar para garantizar el cumplimiento y, al mismo tiempo, preservar la privacidad del cliente. Por ejemplo, un caso de uso interesante son las llamadas credenciales anónimas, donde una parte de confianza puede atestiguar que una credencial de usuario contiene atributos cuyos valores son correctos, lo que permite probar ciertas propiedades en forma de conocimiento cero.
    
    \item Evaluación del riesgo hipotecario. Es posible probar que el salario de una persona está por encima de cierto umbral para que se apruebe una hipoteca. En general, la validación del umbral es una verificación clave que debe realizarse en la evaluación del riesgo financiero. Por lo tanto, ZKRP resulta ser muy importante para las instituciones financieras.
    
    \item Calificación y grado de inversión El problema de calificar empresas según su nivel de productividad o salud financiera se puede modelar determinando una partición de un intervalo numérico, dado por una secuencia de números crecientes $A_{0}, A_{1}, \dots, A_{k}$, tales que la puntuación más alta se atribuye a las empresas calificadas por encima de $A_{k}$ (o, a veces, por debajo de $A_{0}$). La salud de la empresa se mide para obtener un valor $x$, y la nota resultante depende del sub-intervalo al que pertenezca $x$. Por lo tanto, es necesario verificar si $x \in [A_{i}, A_{i+1})$ para cada $0 \leq i \leq k$ . Hasta donde sabemos, no hay investigaciones que apliquen ZKRP a este problema específico, donde tal vez podrían existir construcciones más eficientes, en comparación con la solución sencilla de usar ZKRP $k + 1$ veces.
    
    \item Voto electrónico: Este es un tema importante de investigación, que atrajo la atención de muchos investigadores en los últimos años. Se propusieron diferentes soluciones para diferentes tipos de elecciones. Algunas soluciones se basan en pruebas de conocimiento cero, como ZKRP, prueba de barajado, prueba de descifrado y otras técnicas relacionadas, mientras que otros utilizan diferentes primitivas criptográficas, como el cifrado de umbral homomórfico y la computación multipartita (MPC).
    
    \item Subastas y adquisiciones electrónicas: Las subastas electrónicas seguras es un tema que ha sido foco de investigación durante mucho tiempo, y es una motivación importante en el estudio de ZKRPs, ya que es una de las principales técnicas criptográficas que se pueden utilizar para construir protocolos seguros.
\end{itemize}

\section{Marco histórico}

Las pruebas de conocimiento cero fueron concebidas por primera vez en 1985 por Shafi Goldwasser, Silvio Micali y Charles Rackoff en su artículo ``The Knowledge Complexity of Interactive Proof-Systems'' \cite{Historia}. Este documento introdujo la jerarquía IP de los sistemas de prueba interactivos y concibió el concepto de complejidad del conocimiento, una medida de la cantidad de conocimiento sobre la prueba transferida del probador al verificador. También dieron la primera prueba de conocimiento cero para un problema concreto, el de decidir los no residuos cuadráticos mod m. Junto con un artículo de László Babai y Shlomo Moran, este artículo histórico inventó sistemas de prueba interactivos, por los cuales los cinco autores ganaron el primer Premio Gödel en 1993.

En sus propias palabras, Goldwasser, Micali y Rackoff dicen: \\
 ``De particular interés es el caso donde este conocimiento adicional es esencialmente 0 y mostramos que es posible probar interactivamente que un número es cuadrático sin residuo mod m liberando 0 conocimiento adicional. Esto es sorprendente ya que no se conoce ningún algoritmo eficiente para decidir el mod m de los residuos cuadráticos cuando no se proporciona la factorización de m. Además, todas las pruebas NP conocidas para este problema exhiben la descomposición en factores primos de m. Esto indica que agregar interacción al proceso de prueba puede disminuir la cantidad de conocimiento que se debe comunicar para probar un teorema.''

Oded Goldreich, Silvio Micali y Avi Wigderson llevaron esto un paso más allá y demostraron que, suponiendo la existencia de un cifrado irrompible, se puede crear un sistema de prueba de conocimiento cero para el problema de coloreado de gráficos NP completos con tres colores. Dado que todos los problemas en NP (conjunto de problemas que pueden ser resueltos en tiempo polinómico por una máquina de Turing no determinista) se pueden reducir eficientemente a este problema, esto significa que, bajo esta suposición, todos los problemas en NP tienen pruebas de conocimiento cero. El motivo de la suposición es que, como en el ejemplo anterior, sus protocolos requieren cifrado. Una condición suficiente comúnmente citada para la existencia de un cifrado irrompible es la existencia de funciones unidireccionales, pero es concebible que algunos medios físicos también puedan lograrlo.

Además de esto, también demostraron que el problema de no isomorfismo de gráficos, el complemento del problema de isomorfismo de gráficos, tiene una prueba de conocimiento cero. Este problema está en co-NP, pero actualmente no se sabe que esté en NP ni en ninguna clase práctica. De manera más general, Russell Impagliazzo y Moti Yung, así como Ben-Or et al. continuaría mostrando que, también asumiendo funciones unidireccionales o encriptación indescifrable, que hay pruebas de conocimiento cero para todos los problemas en IP = PSPACE, o en otras palabras, cualquier cosa que pueda probarse mediante un sistema de prueba interactivo puede probarse con cero conocimiento.

\section{Protocolos de conocimiento cero en el contexto docente}

Debido a la importancia de los protocolos de conocimiento cero en los últimos años, han surgido un gran número de artículos que se centran en estudiar sus posibles aplicaciones y los resultados que se pueden conseguir con ellos, como \cite{Bulletproofs} ó \cite{Bulletproofs2}. También han surgido artículos que se centran en el estudio de un nuevo algoritmo de conocimiento cero y comparar su rendimiento y resultados con otros ya existentes previamente, como \cite{Sharp}. E incluso existen algunos que recompilan varios de ellos para indicar las ventajas de unos frente a los otros, como \cite{Survey}.

Sin embargo, no hemos podido encontrar ninguno que se centre en el marco académico, intentando explicar el funcionamiento general de estos protocolos y mostrando ejemplos o proporcionando herramientas para facilitar el entendimiento de su funcionamiento.

Por ejemplo, podemos destacar artículos que han motivado este trabajo, como \cite{Boudot}, que ofrece una explicación en detalle del algoritmo \emph{Square Decomposition}, pero no ofrece pruebas ni herramientas que faciliten su comprensión ni demuestran su funcionamiento; o \cite{Survey}, que proporciona una visión general de los protocolos de conocimiento cero, centrándose en las \emph{Bulletproofs} y sus posibles optimizaciones, pero igualmente sólo ofrece una descripción de su funcionamiento, detallando su algoritmo, pero que puede ser bastante difícil de comprender debido a la complejidad de dicho algoritmo.

Por esta ausencia de herramientas para facilitar la comprensión de los protocolos de conocimiento cero y por la importancia de dichos protocolos especialmente en el marco de la ciberseguridad , surge la motivación de este TFG de proporcionar una explicación sobre su funcionamiento general acompañado por distintas herramientas que intentan ofrecer una manera sencilla de comprender cómo funcionan y cuáles pueden ser sus posibles aplicaciones.

Es por eso que en este trabajo, en lugar de centrarnos en algoritmos que pueden resultar más potentes, decidimos centrarnos en otros más simples y fáciles de comprender, proporcionando una herramienta que podría facilitar su estudio y comprensión.